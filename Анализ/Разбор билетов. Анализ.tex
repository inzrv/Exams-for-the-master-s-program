\documentclass[12pt]{article}
\usepackage[utf8]{inputenc} % Включаем поддержку UTF8
\usepackage[T2A]{fontenc}
\usepackage[russian]{babel}
\usepackage{amsmath}
\usepackage{amsfonts}
\usepackage{verbatim}
\usepackage{amssymb}
\usepackage{amsthm}
\usepackage{caption}
\usepackage{graphicx}%Вставка картинок правильная
\usepackage{float}%"Плавающие" картинки
\usepackage{wrapfig}%Обтекание фигур (таблиц, картинок и прочего)

\begin{document}
\textbf{Вопросы. Математический анализ.} \\



	\textbf{1. (НОД-МСК)} Числовые множества. Грани множеств. Множества в конечномерном действительном пространстве. Соответствие множеств. Счетные и несчетные множества.
	
	\noindent\rule{\textwidth}{1pt}
    
    
    \textbf{Определение:} Наименьшее из чисел, ограничивающих множество $X \subset \mathbb{R}$ сверху, называется \textit{верхней гранью} и обозначается $\sup X$. Или
    $$
    	(s = \sup X) := \forall x \in X ((x \leq s) \wedge (\forall s' < s \, \exists x' \in X (s' < x'))).
    $$
    
   \textbf{ Определение:} \textit{Соответствием} между множествами $A$ и $B$ называют произвольное подмножество декартова произведения 
   $$
   		F \subset A \times B. 
   $$
   
   \textbf{Определение:} \textit{Отображением} из множества $A$ в множество $B$ называется однозначное соответствие между $A$ и $B$, т.е. такое соответствие, что для любого элемента из $A$ найдется ровно один элемент из $B$.
   
   Для отображения используют запись: $F: A \rightarrow B$, для отдельных элементов $b = F(a)$. Отображения также называют \textit{функциями}.
   
  \textbf{Определение:} Множество $X$ называется \textit{счетным}, если оно равномощно множеству $\mathbb{N}$ натуральных чисел, т.е. $|X| = | \mathbb{N}|$. 
  
  \textbf{Определение:} Множество $X$ равномощно множеству $Y$, если существует биективное отображение $X$ на $Y$. 
  
  Про отображение $F: X \rightarrow Y$ говорят, что оно 
  
  \textit{сюръективно}, если $F(X) = Y$; 
  
  \textit{инъективно}, если $ \forall x_1, x_2 \in X \; (F(x_1) = F(x_2)) \Rightarrow (x_1 = x_2)$;
  
  \textit{биективно} (или взаимно однозначно), если оно сюръективно и инъективно одновременно.
  
  \textbf{Определение:} \textit{Несчетное} множество --- бесконечное множество, не являющееся счетным. Множество $\mathbb{R}$ действительных чисел называют \textit{числовым континуумом}, а его мощность --- \textit{мощностью континуума}. 
  
  \textbf{Теорема (Кантор).} Бесконечное множество $\mathbb{R}$ имеет мощность большую, чем бесконечное множество  $\mathbb{N}$. 
  
  $\blacktriangleleft$ Докажем, что отрезок $[0,1]$ больше $\mathbb{N}$. Пусть точки отрезка можно занумеровать $x_1, ..., x_n, ...$ Тогда построим отрезок $I_1$, не содержащий точку $x_1$. Внутри него построим отрезок $I_2$, не содержащий $x_2$, и т. д. Получим последовательность вложенных отрезков, которая по лемме о вложенных отрезках содержит точку $c$. Точка $c$ по построению не совпадает ни с какой точкой последовательности $x_1, ..., x_n, ...$ $\blacktriangleright$ 
   
   \textbf{Условимся} через $ \mathbb{R}^m $ обозначать множество всех упорядоченных наборов $ (x^1, ..., x^m) $, состоящих из $m$ действительных чисел $x^i \in \mathbb{R}$. 
   
   \textbf{Определение:} \textit{Метрикой} или \textit{расстоянием} назовем функцию 
   $$ 
   d:X \times X \rightarrow \mathbb{R}, 
   $$ определенную на парах $ (x_1, x_2) $ точек некоторого множества $X$ и обладающую следующими свойствами:
   \begin{enumerate}
    \item $d(x_1, x_2) \geq 0$;
    \item $d(x_1, x_2) = 0 \Leftrightarrow x_1 = x_2$;
    \item $d(x_1, x_2) = d(x_2, x_1)$;
    \item $d(x_1, x_3) \leq d(x_1, x_2) + d(x_2, x_3)$.
   \end{enumerate}
   
   Множество $X$ вместе с фиксированной функцией $d$ называют\textit{ метрическим пространством}.
   
   Пример метрики $d: \mathbb{R}^m \times \mathbb{R}^m  \rightarrow \mathbb{R}$:
   $$
   	d(x_1, x_2) = \sqrt{\sum_{i=1}^m (x_1^i - x_2^i)^2}.
   $$
   
   
  \textbf{ Определение:} При $\delta > 0 $ множество 
  $$
  	B(a, \delta) = \{ x \in \mathbb{R}^m \; | \;  d(a,x) < \delta \}
  $$
   называется \textit{шаром} с центром $a \in \mathbb{R}^m$ радиуса $\delta$ или $\delta$-\textit{окрестностью} точки  $a \in \mathbb{R}^m$.
   
   \textbf{Определение:} Множество $G \in \mathbb{R}^m$ называется \textit{открытым} если для любой точки $x \in G$ существует шар $B(x,\delta)$ такой, что $B(x, \delta) \subset G$. Например, шар $B(a,r)$ --- открытое множество в $\mathbb{R}^m$. 
   
    \textbf{Определение:} Множество $F \in \mathbb{R}^m$ называется \textit{замкнутым}, если его дополнение $G = \mathbb{R}^m \setminus F$ является открытым. Пример: \textit{сфера} $S(a,r) = \{ x\in \mathbb{R}^m \; | \; d(x, a) = r\}, \; r \geq 0$.
    
   \textbf{Определение:} Открытое в $\mathbb{R}^m$ множество, содержащее некоторую точку, называется окрестностью данной точки в $\mathbb{R}^m$.
    
   \textbf{Определение:} Точка $x \in \mathbb{R}^m $ по отношению к множеству $E \in \mathbb{R}^m$ называется \\
  \textit{внутренней точкой} $E$, если она содержится в $E$ вместе с некоторой своей окрестностью; \\
  \textit{внешней точкой} $E$, если она является внутренней точкой дополнения к $E$ в $\mathbb{R}^m$. \\
  \textit{граничной точкой} $E$, если она не является ни внутренней, ни внешней точкой $E$.
  
  \textbf{Определение:} Точка $a \in \mathbb{R}^m$ называется \textit{предельной} точкой  множества $E$, если для любой окрестности $O(a)$ точки $a$ пересечение $E \cap O(a)$ есть бесконечное множество. 
  
 \textbf{Определение:} Объединение множества $E$ и всех его предельных точек из $\mathbb{R}^m$ называется \textit{замыканием} множества $E$ в $\mathbb{R}^m$ (обозначение символом $\overline{E}$).
 
 Пример: Множество $\overline{B}(a,r) = B(a,r) \cup S(a,r)$ есть множество предельных точек для шара $B(a,r)$, поэтому $\overline{B}(a,r)$ --- замкнутый шар.
 
 \textbf{Определение:} Множество $K \subset \mathbb{R}^m$ называется \textit{компактом}, если из любого покрытия $K$ открытыми в $\mathbb{R}^m$ множествами можно выделить конечное покрытие. 
 
 Пример: Обобщением отрезка в $\mathbb{R}^m$ является множество
 $$	
 	I = \{ x \in \mathbb{R}^m \; | \; a^i \leq x \leq b^i, \; i = 1,...,m \}.
 $$ 
 \textbf{Утверждение} о том, что $I$ --- компакт, доказывается аналогично лемме о конечном покрытии. Если $I$ нельзя покрыть конечным набором множеств, то его стороны можно разделить пополам, и хотя бы одну из $2^m$ частей нельзя покрыть конечным набором открытых множеств. Тогда повторяем для нее эту операцию. Далее, по лемме об общей  точке системы вложенных отрезков, существует точка $c \in I$, которая принадлежит всем вложенным промежуткам. Тогда ее покрывает некоторое множество $G$, но тогда $G$ покрывает все промежутки, начиная с некоторого номера $n$. Противоречие.


\textbf{Утверждение:} Если $K$ --- компакт, то $K$ --- замкнутое множество и любое замкнутое множество в $K$ само является компактом. 

\textbf{Определение:} Диаметром множества $E \in \mathbb{R}^m$ называется величина
$$	
	d(E) := \sup_{x_1, x_2 \in E} d(x_1, x_2).
$$

\textbf{Определение:} Множество $E \subset \mathbb{R}^m$ называется \textit{ограниченным}, если его диаметр конечен.


\textbf{Утверждения:} $K$ --- компакт $\Rightarrow$ $K$ --- ограниченное подмножество $\mathbb{R}^m$. $K$ --- компакт $\Leftrightarrow$ $K$ ограничено и замкнуто. 

\noindent\rule{\textwidth}{1pt}
\\


\textbf{2. (НОД-МСК, (3 SE))} Числовые последовательности и пределы. Свойства сходящихся последовательностей. Признаки существования предела. Первый и второй замечательные пределы. 

\noindent\rule{\textwidth}{1pt}

\textbf{Определение:} Функция $f: \mathbb{N} \rightarrow \mathbb{R}$ называется \textit{числовой последовательностью}.

\textbf{Определение:} Число $A \in \mathbb{R}$ называется пределом числовой последовательности $\{ x_n \}$, если для любой окрестности $V(A)$ точки $A$ существует такой номер $N$, что все члены последовательности, номера которых больше $N$ содержатся в указанной окрестности.

Через $\varepsilon$:

$$
	(\lim_{n \rightarrow \infty} x_n = A) := \forall \varepsilon > 0 \, \exists N \in \mathbb{N} \; \forall n > N \, (|x_n - A| < \varepsilon).
$$

\textbf{Свойства предела последовательности.}

\textbf{Определение:} Последовательность называется \textit{ограниченной}, если существует число $M$ такое, что $|x_n| < M$ для любого $n \in \mathbb{N}$.


\textbf{Теорема.} a) Финально постоянная последовательность сходится.

b) Любая окрестность предела последовательности содержит все члены последовательности, за исключением конечного их числа.

c) Последовательность не может иметь двух различных пределов.

d) Сходящаяся последовательность ограничена. \\

\textbf{Предельный переход и арифметические операции.}

\textbf{Теорема.} Если $ \{ x_n \} \rightarrow A \in \mathbb{R}$, $\{ y_n \} \longrightarrow B \in \mathbb{R}$, то

a) $\lim_{n\rightarrow \infty} (x_n + y_n) = A + B$;

b) $\lim_{n\rightarrow \infty} (x_n \cdot y_n) = A \cdot B$;

c) $\lim_{n\rightarrow \infty} \frac{x_n}{y_n} = \frac{A}{B}$, если $y_n \neq 0, \; B \neq 0$. \\

\textbf{Предельный переход и неравенства.}

\textbf{Теорема.} a) Пусть $ \{ x_n \} \rightarrow A \in \mathbb{R}$, $\{ y_n \} \longrightarrow B \in \mathbb{R}$. Если $A < B$, то $\exists N \in \mathbb{N}$ такой, что при любом $n  > N$ выполнено неравенство $x_n < y_n$.

b) Пусть последовательности $\{x_n\}$, $\{y_n\}$, $\{z_n\}$ таковы, что найдется номер $N \in \mathbb{N}$ такой, что для всех $n  > N$ выполняются неравенства $x_n \leq y_n \leq z_n$. При этом последовательности $\{x_n\}$, $\{y_n\}$ сходятся к одному пределу. Тогда и $\{ y_n \}$ сходится к этому пределу. \\


\textbf{Вопросы существования предела последовательности.}

\textbf{Определение:} Последовательность называется фундаментальной (или последовательностью Коши), если для любого числа $\varepsilon > 0$ существует $N \in \mathbb{N}$, что из $n,m > N$ следует, что $|x_n - x_m| < \varepsilon$. 

\textbf{Теорема. Критерий Коши сходимости последовательности.} Последовательность сходится тогда и только тогда, когда она фундаментальна.

$\blacktriangleleft$ $(\Rightarrow) \; |x_n - A| < \varepsilon / 2 $, $|x_m - A| < \varepsilon / 2$, тогда $|x_n - x_m| \leq |x_n - A| + |x_m - A| < \varepsilon$.

$(\Leftarrow)$ Для некоторого $N$ имеем 
$$
x_N - \frac{\varepsilon}{3} < x_k < x_N + \frac{\varepsilon}{3}.
$$
Пусть $a := \inf x_k$, $b := \sup x_k$, тогда последовательность вложенных отрезков $[a_k; b_k]$ имеет общую точку $A$. Тогда 
$|A - x_k| \leq b_k - a_k$, но $x_N - \frac{\varepsilon}{3} \leq a_k, \; x_N + \frac{\varepsilon}{3} \geq b_k$, значит 
$|x_k - A| \leq \frac{2\varepsilon}{3} < \varepsilon$.
$\blacktriangleright$ \\

\textbf{Определение:} Последовательность $\{ x_n\}$ называется \textit{возрастающей}, если $x_n > x_{n-1}$ для любого $n$. $\{ x_n \}$ --- \textit{неубывающая} последовательность, если $x_n \geq x_{n-1}$ для любого $n$. \textit{Убывающая} и \textit{невозрастаюшая} определяются аналогично.


\textbf{Теорема (Вейерштрасс).} Для того чтобы неубывающая последовательность имела предел, необходимо и достаточно, чтобы она была ограничена сверху. 

$\blacktriangleleft$ То, что всякая сходящаяся последовательность ограничена было доказано выше. Докажем в обратную сторону. Рассмотрим $s := \sup x_n$, который существует в силу того, что множество принимаемых значений ограничено сверху. По определению супремума $\forall \varepsilon > 0 \; \exists N \; x_N > s - \varepsilon$. Тогда $\forall n > N \;  s- \varepsilon < x_N \leq x_n \leq s$, то есть $|s - x_n| = s - x_n < \varepsilon$.
$\blacktriangleright$ \\ 


\textbf{Подпоследовательность и частичный предел.}

\textbf{Определение:} Если $x_1, \, x_2, \, ... \, $ --- некоторая последовательность, а $n_1 < n_2 < ...$ --- возрастающая последовательность натуральных чисел, то последовательность $x_{n_1}, \,x_{n_2}, ...$ называется \textit{подпоследовательностью} последовательности $\{x_n\}$. 

\textbf{Теорема Больцано-Вейерштрасса.} Каждая ограниченная последовательность действительных чисел содержит сходящуюся подпоследовательность. 

$\blacktriangleleft$
Если множество $E$ значений ограниченной последовательности конечно, то существует хотя бы один элемент $e \in E$ и последовательность номеров $ n_1 < n_2<...$ такие, что $x_{n_1} = x_{n_2} = ... = e$. 

Если множество $E$ бесконечно, то по принципу Больцано-Вейерштрасса оно обладает хотя бы одной предельной точкой $e$. Далее выбираем номера $n_k$ такие, что $|x_{n_k} - e| < \frac{1}{k}$. Поскольку $\{  \frac{1}{k} \} \rightarrow 0$, то полученная подпоследовательность сходится к $e$.
$\blacktriangleright$

\textbf{Лемма Больцано-Вейерштрасса.} Всякое бесконечное ограниченное множество содержит предельную точку (предельная точка множества $X$ --- это точка, любая окрестность которой содержит бесконечное подмножество множества $X$).


$\blacktriangleleft$
Докажем для $X \subset \mathbb{R}$. Из определения ограниченности следует, что $X$ содержится в некотором отрезке $ I =[a;b]$. Докажем, что хотя бы одна точка $I$ является предельной. 

Предположим, что это не так. Тогда каждая точка из $x \in I$ имела бы окрестность $U(x)$, содержащую лишь конечное число точек из $X$. По лемме о конечном покрытии из совокупности этих окрестностей можно выделить конечное покрытие отрезка $I$, а значит и конечное покрытие множества $X$.Так как количество окрестностей в покрытии конечно и каждая окрестность содержит лишь конечное число точек из $X$, то во всем покрытии конечное число точек, что противоречит тому, что $X$ бесконечное множество.
$\blacktriangleright$ \\

\textbf{Первый замечательный предел.} Докажем, что
$$
	\lim_{x\rightarrow 0} \frac{\sin x}{x} = 1.
$$
$\blacktriangleleft$ a) Покажем, что 
$\cos^2 x < \frac{\sin x}{x} < 1$ при $0 < |x| < \frac{\pi}{2}$. Из рис. 1 и определения косинуса и синуса, сравнивая площади сектора $\sphericalangle OCD$, треугольника $\vartriangle OAB$ и сектора $\sphericalangle OAB$, имеем

$$
	S_{\sphericalangle OCD} = \frac{1}{2} x\cos^2{x} < 
	S_{\vartriangle OAB} = \frac{1}{2} \sin{x} < S_{\sphericalangle OAB} = \frac{1}{2} x.
$$

Получаем $x \cos^2{x} < \sin{x} < x$ или $1 - \sin^2{x} < \frac{\sin{x}}{x} < 1$.

b) Из a) следует, что $\sin{x} \leq x$ при любом $x \in \mathbb{R}$. Тогда $\lim_{x\rightarrow 0}\sin{x} = 0$.

c) $\lim_{x\rightarrow 0} (1 - \sin^2{x}) = 1 -\lim_{x \rightarrow 0}\sin^2{x} = 1 - \lim_{x\rightarrow 0} \sin{x} \cdot\lim_{x\rightarrow 0} \sin{x} = 1$. Значит из a) и теореме о предельном переходе в неравенствах можем заключить, что
$$
	\lim_{x\rightarrow 0} \frac{\sin{x}}{x} = 1. \blacktriangleright
$$


\begin{figure}

\centering

\includegraphics[width=0.3\linewidth]{pervii.png}

\caption{Конструкция для первого замечательного предела}

\label{fig:mpr}

\end{figure}


\textbf{Второй замечательный предел.}

Докажем, что 
$$
\lim_{x\rightarrow \infty} \left( 1 +\frac{1}{x} \right)^x = e,
$$
где $e$ --- некоторое вещественное число.

$\blacktriangleleft$

Для начала докажем существование предела $\{ (1 + \frac{1}{n})^n\}$. Для этого заметим, что при любых $n \in \mathbb{N}$ и $\alpha \geq -1$ выполняется неравенство Бернулли:

$$
(1 + \alpha) ^ n \geq 1 + n\alpha. 
$$
Доказывается индукцией по $n$.

Теперь покажем, что последовательность $x_n = \left( 1 + \frac{1}{n} \right)^{n+1} $ --- убывающая. 

При $n \geq 2$ находим

$$
\frac{x_{n-1}}{x_{n}} = \frac{n^{2n}}{(n^2-1)^n} \frac{n}{n+1} = 
\left( 1 + \frac{1}{n^2 - 1}\right)^n \frac{n}{n+1} \geq \left( 1 + \frac{n}{n^2 - 1}\right) \frac{n}{n+1} > 1.
$$
Таким образом последовательность убывает и ограничена снизу 0, а значит имеет предел. Легко показать, что тогда и последовательность $ \{ \left( 1 + \frac{1}{n} \right)^{n} \}$ имеет предел.


Теперь осталось доказать, что $\lim_{x \rightarrow \infty} \left( 1 + \frac{1}{x} \right)^x = e$.

Рассмотрим случай $x \rightarrow \infty$. Пусть $n = [x]$ --- целая часть $x$, тогда $n \leq x < n+1$. Тогда
$$
	\frac{1}{n+1} < \frac{1}{x} \leq \frac{1}{n} \Rightarrow 
	\left( 1+ \frac{1}{n+1} \right)^n < \left( 1 + \frac{1}{x}\right)^x \leq \left( 1 + \frac{1}{n} \right)^{n+1}.
$$
Далее заметим, что
$$
\lim_{n \rightarrow \infty} \left( 1+ \frac{1}{n+1} \right)^n = 
\lim_{n\rightarrow \infty} \frac{\left( 1 + \frac{1}{n+1} \right)^{n+1}}{\left( 1 + \frac{1}{n+1} \right)} = \frac{e}{1} = e;
$$

$$
\lim_{n \rightarrow \infty} \left( 1 + \frac{1}{n} \right)^{n+1} = 
\lim_{n \rightarrow \infty} \left( 1 + \frac{1}{n} \right)^{n} \cdot 
\lim_{n \rightarrow \infty} \left( 1 + \frac{1}{n} \right) = e \cdot 1 = e.
$$
По теореме о двух полицейских получаем искомый результат.
$\blacktriangleright$ \\

\noindent\rule{\textwidth}{1pt}
\textbf{4. (SE)} Два определения предела функции одной и нескольких переменных: с помощью окрестностей и через пределы последовательностей.
\noindent\rule{\textwidth}{1pt}


Пусть $E$ --- некоторое подмножество множества $\mathbb{R}$ действительных чисел и $a$ --- предельная точка множества $E$ (т.е. такая, что в любой ее окрестности содержится бесконечное подмножество $E$). 

\textbf{Определение (по Коши).} Функция $f : E \rightarrow \mathbb{R}$ стремится к $A$ при $x$, стремящимся к $a$, если для любого $\varepsilon > 0$ существует число $\delta > 0$ такое, что 	для любой точки $x \in E$ такой, что $0< |x - a| < \delta$, выполнено соотношение $|f(x) - A| < \varepsilon$. 

В логической символике

$$
\forall \varepsilon > 0 \; \exists \delta > 0 \; \forall x \in E \; (0 < |x -a| < \delta \Rightarrow |f(x) - A| < \varepsilon).  
$$

\textbf{Напоминание:} Окрестностью точки в  $\mathbb{R}^m$ называют любое открытое множество, содержащее данную точку. Для $m=1$ окрестность точки --- это любой интервал, содержащий данную точку. \textit{Проколотая} окрестность точки --- это окрестность, из которой удалена сама эта точка. Множества 
$$
	U_E(a) := U(a) \cap E, \; \mathring{U}_E(a) := \mathring{U}(a) \cap E
$$
будем называть окрестностью и проколотой окрестностью точки $a$ в множестве $E$.


\textbf{Определение предела функции через окрестности.}

$$
\left( \lim_{x\rightarrow a}f(x) = A \right) := 
\forall V_{\mathbb{R}}(A) \; \exists \mathring{U}_E(a) \; \left( f(\mathring{U}_E(a)) \subset V_{\mathbb{R}}(A)  \right)
$$


\textbf{Утверждение. Определение предела функции через пределы последовательностей (определение по Гейне).} Соотношение $\lim_{E \ni x \rightarrow a} f(x) = A$ имеет место тогда и только тогда, когда для любой последовательности $\{ x_n \}$ точек из $E \setminus a$, сходящейся к $a$, последовательность $\{ f(x_n) \}$ сходится к $A$. \\



\textbf{Предел функции нескольких переменных.}


Если каждому натуральному числу $n$ поставлена в соответствие точка $x_n \in \mathbb{R}^m$, то говорят, что задана последовательность точек $\{ x_n\}$ в пространстве $\mathbb{R}^m$.

 
\textbf{Определение.} Точка $A \in \mathbb{R}^m$ называется пределом последовательности $\{ x_n \}$ если 
$$
\lim_{n \rightarrow \infty} d(A, x_n) = 0.
$$

\textbf{Лемма.} Последовательность точек $x_n(x_1^{(n)}, ..., x_m^{(n)})$ сходится к точке $A(a_1,...,a_m)$ тогда и только тогда, когда последовательности $\{ x_i^{(n)}\}$ сходятся к соответствующим координатам $a_i$ точки $A$. 

$\blacktriangleleft$
Утверждение леммы следует из формулы для расстояния $$d(x_1,x_2): \mathbb{R}^m \times \mathbb{R}^m \rightarrow \mathbb{R}.\blacktriangleright$$


\textbf{Определение.} Последовательность точек $\{ x_n \in \mathbb{R}^m \}$ называется \textit{фундаментальной}, если
$$
	\forall \varepsilon > 0 \; \exists N \in \mathbb{N} \; \forall n,m > N \; d(x_n, x_m) < \varepsilon.
$$
Для того, чтобы последовательность была фундаментальной необходимо и достаточно, чтобы фундаментальными были все последовательности $\{ x_i^{(n)}\}$.

\textbf{Теорема. Критерий Коши сходимости последовательности.} Последовательность $\{ x_n \} $ точек из $\mathbb{R}^m$ сходится тогда и только тогда, когда она фундаментальна.

\textbf{Определение.} Пусть $ \{ M (x_1,..., x_m)\}$ --- множество точек из $\mathbb{R}^m$ и пусть каждой точке $M$ поставлено в соответствие некоторое число $u$. Тогда говорят, что на множестве $\{ M \}$ \textit{определена функция $m$ переменных} (см. Рис. 2).


\begin{figure}
\centering
\includegraphics[width=0.5\linewidth]{fmp.png}
\caption{График функции двух переменных}
\label{fig:mpr}
\end{figure}

Пусть $A$ --- предельная точка множества $\{ M \}$,а функция $u = f(M)$ определена на множестве $\{ M \}$.

\textbf{Определение предела функции нескольких переменных (по Коши).} Число $b$ называется \textit{пределом} функции $u=f(M)$ в точке $A$ (при $M \rightarrow A$), если для любого $\varepsilon > 0$ существует $\delta > 0$ такое, что для любой точки $M \in \{ M \}$, удовлетворяющей условию $0 < d(A, M) < \delta$, выполняется неравенство $|b - f(M)| < \varepsilon$.


\textbf{Определение предела функции нескольких переменных (по Гейне).} Число $b$ называется \textit{пределом} функции $u = f(M)$ в точке $A$, если для любой последовательности точек из $M$, сходящейся к $A$, соответствующая последовательность $\{ f(M)\}$ сходится к $b$. \\


\noindent\rule{\textwidth}{1pt}
\textbf{4,5. Дополнительно.} Непрерывность функции многих переменных.

\noindent\rule{\textwidth}{1pt}


Пусть функция $u = f(M)$ определена на множестве $\{ M\} \subset \mathbb{R}^m$ и пусть точка $A \in \{ M \}$ --- предельная точка множетсва $\{ M \}$.

\textbf{Определение.} Функция $u = f(M)$ называется \textit{непрерывной в точке} $A$, если 
$$
	\lim_{M \rightarrow A}f(M) = f(A).
$$


\textit{Точка разрыва} функции $f(M)$ --- это предельная точка множества $M$, в которой функция не является непрерывной. 

\textbf{Определение.} Приращением (полным приращением) функции $u = f(M)$ в точке $A$ называется функция $\Delta u = f(M) - f(A)$.

Условие непрерывности функции в точке $A$ можно переписать в виде
$$
	\lim_{M \rightarrow A}\Delta u = \lim_{M \rightarrow A}[f(M) - f(A)] = 0.
$$
Такое равенство называется\textit{ разностной формой условия непрерывности функции в точке} $A$.


Если $M=(x_1, ..., x_m)$, $A = (a_1, ..., a_m)$, $\Delta x_i = x_i - a_i$, то разностная форма условия непрерывности принимает вид
$$
	\lim_{\Delta x_i \rightarrow 0 } \Delta u = 0.
$$

\textbf{Определение.} \textit{Частным приращением функции} $f(x,y)$ в точке $M_0$. Называется функция одной переменной $\Delta x$ вида
$$
	\Delta_x u = f(x_0 + \Delta x, y_0) - f(x_0, y_0).
$$


\textbf{Определение.} Функция $u = f(x,y)$ называется непрерывной в точке $M_0(x_0,y_0)$ по переменной $x$, если 
$$
	\lim_{\Delta x \rightarrow 0} \Delta_x u = 0.
$$


\textbf{Аналогичное определение.} Функция $u = f(x,y)$ называется \textit{непрерывной в точке $M_0(x_0,y_0)$ по переменной} $x$, если при фиксированном значении переменной $y = y_0$ предел функции $f(x,y_0)$ одной переменной $x$ существует и равен
$$
\lim_{x\rightarrow 0}f(x,y_0) = f(x_0,y_0).
$$

\textbf{Теорема.} Если функция $f(x,y)$ определена в окрестности точки $M_0$ и непрерывна в $M_0$, то она непрерывна в ней по отдельным переменным. Обратное в общем случае неверно (функция может быть непрерывна в точке по отдельным переменным, но быть разрывной по совокупности переменных).

\noindent\rule{\textwidth}{1pt}
\textbf{5. (SE)} Производные и дифференциалы функции одной и нескольких переменных.\\
\noindent\rule{\textwidth}{1pt}


Пусть функция $y = f(x)$ определена на интервале $(a,b)$. Зафиксируем точку $x$ из $(a,b)$ и рассмотрим другую точку $x + \Delta x$. Величину $\Delta x$ назовем \textit{приращением аргумента} в точке $x$. Пусть
$$
	\Delta y = f(x + \Delta x) - f(x),
$$
при фиксированном $x$ эта разность является функцией от $\Delta x$ и называется \textit{приращением функции} $f(x)$ в точке $x$.

Рассмотрим отношение
$$
\frac{\Delta y}{\Delta x} = \frac{f(x + \Delta x) - f(x)}{\Delta x},
$$
которое также является функцией аргумента $\Delta x$. 

\textbf{Определение.} Если существует 
$$
	\lim_{\Delta x \rightarrow 0}   \frac{\Delta y}{\Delta x}, 
$$
то он называется \textit{производной функции $y = f(x)$ в точке $x$}.

\textbf{Пример.} Рассмотрим функцию $f(x) = \sin{x}$. Для нее 
$$
\Delta y = \sin{(x + \Delta x)} - \sin{x}.
$$

Используем формулу разности синусов, получаем
$$
	\Delta y = 2 \sin \frac{\Delta x}{2} \cdot \cos(x + \frac{\Delta x} {2}).
$$
Нетрудно показать, что $\lim_{\Delta x \rightarrow 0} \cos(x + \Delta x) = \cos x $. Используем это далее, получим
$$
	\lim_{\Delta x \rightarrow 0} \frac{ 2 \sin \frac{\Delta x}{2} \cdot \cos(x + \frac{\Delta x} {2})}{\Delta x} = 
	\lim_{\Delta x \rightarrow 0} \frac{ 2 \sin \frac{\Delta x}{2}}{\Delta x} \cos(x + \frac{\Delta x}{2}) = \cos x.
$$

Таким образом, производная функции $\sin{x}$ есть $\cos x$. \\


\textbf{Дифференцируемость и дифференциал функции.} 
Пусть функция $f(x)$ имеет производную в точке $x$, то есть существует предел
$$
	\lim_{\Delta x \rightarrow 0} \frac{\Delta y}{\Delta x} = f'(x).
$$

Введем функцию 
$$
\alpha(\Delta x) = \frac{\Delta y}{\Delta x} - f'(x),
$$
которая определена при $\Delta x \neq 0$ и является бесконечно малой при $\Delta x \rightarrow 0$. Тогда приращение функции можно расписать как 
$$
	\Delta y = f'(x) \Delta x + \alpha(\Delta x)\Delta x.
$$
Удобно определить $\alpha(0) = 0$ до непрерывности.

Пусть теперь приращение функции можно представить в виде
$$
	\Delta y = A\Delta x + \alpha(\Delta x) \Delta x, 	\eqno(5.1)
$$
где $A$ --- некоторое число, $\alpha(\Delta x) \rightarrow 0$ при $\Delta x \rightarrow 0$, $\alpha(0) = 0$. Тогда
$$
	\lim_{\Delta x  \rightarrow 0} \frac{\Delta y}{\Delta x} = \lim_{\Delta x \rightarrow 0} (A + \alpha(\Delta x)) = f'(x) = A. 
$$
\fbox {
    \parbox{\linewidth}{
    Таким образом, производная функции $f(x)$ в точке $x$ 		существует тогда и только тогда, когда $f(\Delta x) = A \Delta x + \alpha(\Delta x) \Delta x$, где $A \in \mathbb{R}$, $\alpha(\Delta x) \rightarrow 0$, при $\Delta x \rightarrow 0$, $\alpha(0) = 0$.
    }
}\\

\textbf{Определение.} Если приращение функции $f(x)$ в точке $x$ можно представить в виде (5.1), то функция $f(x)$ называется \textit{дифференцируемой} в точке $x$.

Заметим, что для дифференцируемости функции в точке необходимо и достаточно того, чтобы у нее существовала производная в этой точке.

\textbf{Пример.} Рассмотрим функцию $y = x^2$ и докажем, что она дифференцируема в любой фиксированной точке $x \in \mathbb{R}$. Действительно,
$$
\Delta y = (x + \Delta x)^2 - x^2 = 2x \cdot \Delta x + \Delta x \cdot \Delta x,
$$
тогда $\alpha(\Delta x) = \Delta x \rightarrow 0$, $A = 2x$ --- число, не зависящее от $\Delta x$. 



\textbf{Теорема.} Если функция $y = f(x)$ дифференцируема в точке $x$, то она непрерывна в точке $x$. 

$\blacktriangleleft$
Необходимо показать, что если 
$$
\Delta y = f(x + \Delta x) - f(x) = f'(x)\Delta x + o(\Delta x),
\eqno(5.2)
$$

то 
$$
\lim_{\Delta x \rightarrow 0 } f(x + \Delta x) = f(x).
$$

Перепишем (5.2) виде 
$$
f(x + \Delta x) = f(x) + f'(x)\Delta x + o(\Delta x),
$$
и устремим $\Delta x \rightarrow 0$. Тогда
$$
	\lim_{\Delta x \rightarrow 0} f(x + \Delta x) = \lim_{\Delta x \rightarrow 0} (f(x) + f'(x)\Delta x + o(\Delta x)) = \lim_{\Delta x \rightarrow 0}f(x) + 0 + 0 = f(x).
$$
Стоит отметить, что непрерывность функции в точке не означает существование производной в этой точке. Пример: $f(x) = |x|$, которая непрерывна в точке $0$, но не имеет в ней производной. 
$\blacktriangleright$

\textbf{Определение.} Дифференциалом функции $y = f(x)$ в точке $x$ называется линейная функция аргумента $\Delta x$:
$$
	dy = f'(x)\Delta x.
$$
Если $f'(x) \neq 0$, то $ dy = f'(x)\Delta x $ является главной частью $\Delta y$ при $\Delta x \rightarrow 0$. Иначе, не является. Дифференциал независимой переменной определим как $dx = \Delta x$.


\textbf{Физический смысл дифференциала.}  $dy = f'(x)\Delta x = v(x) \Delta x$, то есть дифференциал равен тому изменению координаты, которое имела бы точка, если бы ее скорость $v(x)$ на отрезке времени $[x, x + \Delta x]$ была бы постоянной, равной $f'(x)$.

\textbf{Геометрический смысл дифференциала.} Дифференциал $dy$ равен изменению касательной к графику функции $y = f(x)$ в точке $x$ на отрезке $[x, x + \Delta x]$ (см. Рис. 3). 


\begin{figure}[h]
\centering
\includegraphics[width=0.5\linewidth]{dy.png}
\caption{Геометрический смысл дифференциала функции в точке $x$}
\label{fig:mpr}
\end{figure}

\newpage

\textbf{Частные производные и дифференцируемость функции нескольких переменных.} 

Пусть $M(x_1,...,x_m)$ --- внутренняя точка области определения функции $u = f(M)=f(x_1,..., x_m)$. Рассмотрим частное приращение функции в этой точке:
$$
	\Delta_{x_k}u = f(x_1, ..., x_k + \Delta x_k, ..., x_m)- f(x_1, ..., x_m),
$$ 
которое зависит только от $\Delta x_k$ при фиксированной точке $M$. 


\textbf{Определение.} Если существует 
$$
	\lim_{\Delta x_k \rightarrow 0} \frac{\Delta_{x_k}u}{\Delta x_k}, 
$$
то он называется частной производной функции $u$ в точке $M$. Обозначение: $\frac{\partial u}{\partial_{x_k}}(M)$.

\textbf{Физический смысл частной производной.} Частная производная $\frac{\partial u}{\partial x_k}$ характеризует скорость изменения функции в точке в направлении оси $Ox$.


Рассмотрим теперь полное приращение $\Delta u$  функции $u = f(x_1,...,x_m)$ во внутренней точке $M(x_1, ..., x_m)$ из области определения функции:
$$
	\Delta u = f(x_1 + \Delta_{x_1}, ..., x_m + \Delta_{x_m}) - f(x_1, ..., x_m).
$$

\textbf{Определение.} Функция $f(x_1, ..., x_m)$ называется \textit{дифференцируемой в точке $M(x_1, ..., x_m)$}, если ее полное приращение в этой точке можно представить в виде 

$$
 \Delta u = A_1\Delta x_1 + ... + A_m \Delta x_m + \alpha_1 \Delta x_1 + ... + \alpha_m \Delta x_m, \eqno(5.3)
$$
где $A_i$ --- какие-то числа, не зависящие от $\Delta x_i$, $\alpha_i = \alpha(\Delta x_1,...,\Delta x_m)$ --- бесконечно малые функции при $\Delta x_1 \rightarrow 0,..., \Delta x_m \rightarrow 0$, равные нулю при $\Delta x_1 = ... = \Delta x_m = 0$.

Условие (5.3) будем называть\textit{ условием дифференцируемости функции в точке}.

Условие (5.3) можно переписать в виде
$$
	\Delta u = A_1\Delta x_1 + ... + A_m \Delta x_m + o(\rho),
$$
где $\rho$ --- расстояние между точками $M(x_1,...,x_m)$ и $M'(x_1 + \Delta x_1, ..., x_m + \Delta x_m)$.

\begin{figure}[h]
\centering
\includegraphics[width=0.5\linewidth]{rho.png}
\caption{Расстояние между точками $M$ и $M'$}
\label{fig:mpr}
\end{figure}
  
  \textbf{Замечание.} Из условия (5.3) следует, что если функция дифференцируема в точке, то она непрерывна в данной точке. Так как при $\Delta u = A_1\Delta x_1 + ... + A_m \Delta x_m + \alpha_1 \Delta x_1 + ... + \alpha_m \Delta x_m $ предел
  $$
  	\lim_{\Delta x_i \rightarrow 0 } \Delta u = 0.
  $$ \\
  
  
\noindent\rule{\textwidth}{1pt}
\textbf{6. (YA)} Необходимое и достаточное условия дифференцируемости функции. Частные производные. Полный дифференциал. Дифференцирование сложной функции.\\
\noindent\rule{\textwidth}{1pt}
  
  
\textbf{Теорема. Необходимое и достаточное условие дифференцируемости функции.} Функция одной переменной $f(x)$ дифференцируема в точке $x_0$ тогда и только тогда, когда существует производная $f'(x_0)$.

$\blacktriangleleft$ 1. ($f$ --- дифференцируема $\Rightarrow$ $\exists f'(x_0)$) Если $f$ --- дифференцируема в $x_0$, тогда ее приращение можно представить в виде 
$$
	\Delta f = A\Delta x + \alpha(\Delta x) \Delta x,
$$
где $\alpha(\Delta x) \rightarrow 0$ при $\Delta x \rightarrow 0$, $A$ --- некоторое число.
Значит, 
$$
	\frac{\Delta f}{\Delta x} = \frac{A\Delta x + \alpha(\Delta x) \Delta x}{\Delta x} = A + \alpha(\Delta x),
$$
тогда
$$
	\lim_{\Delta x \rightarrow 0} \frac{\Delta f}{\Delta x} = \lim_{\Delta x \rightarrow 0} A + \alpha(\Delta x) = A.
$$
Следовательно, существует производная. 


2. ( $\exists f'(x_0)$ $\Rightarrow$ $f$ --- дифференцируема)

По определению производной существует предел
$$
	\lim_{\Delta x \rightarrow 0} \frac{\Delta f}{\Delta x} = A,
$$
что равносильно записи
$$
	\lim_{\Delta x \rightarrow 0} \frac{\Delta f}{\Delta x} - A = 0,
$$
то есть функция $\alpha(x) = \frac{\Delta f}{\Delta x} - A$ --- б.м.ф. при $\Delta x \rightarrow 0$. Тогда приращение функции $f(x)$ в точке $x_0$ можно представить как
$$
	\Delta f = A \Delta x + \alpha(\Delta x) \Delta x, 
$$
где $\alpha(\Delta x) \rightarrow 0$ при $\Delta x \rightarrow 0$, а $A = f'(x_0)$ --- некоторое число. Получаем определение дифференцируемой в точке функции. $\blacktriangleright$

Для функции многих переменных существование частных производных в точке $M_0$ уже не является достаточным условием ее дифференцируемости в этой точке.

\textbf{Теорема.} Если функция $u = f(x_1,...,x_m)$ дифференцируема в точке $M$, то она имеет в точке $M$ частные производные по всем переменным.

$\blacktriangleleft$ Пусть функция $u$ дифференцируема в точке $M$, тогда ее приращение можно записать в виде
$$
	\Delta u = A_1\Delta x_1 + ...+ A_m \Delta x_m + \alpha_1 \Delta x_1 + ... + \alpha_m \Delta x_m,
$$
где $A_i$ --- некоторые числа, $\alpha_i = \alpha_i(\Delta x_1, ..., \Delta x_m) \rightarrow 0$ при $ \forall \Delta x_i \rightarrow 0$, $\alpha_i = 0$ при $\Delta x_1 = ... = \Delta x_m = 0$. Зафиксируем номер $k \in \{1,...,m \}$. Пусть теперь $\Delta x_i = 0$ при $i \neq k$. Тогда
$$
	\Delta u = A_k \Delta x_k + \alpha_k \Delta x_k.
$$
Значит 
$$
	\frac{\partial u}{\partial x_k}(M) = \lim_{\Delta x_k \rightarrow 0} \frac{\Delta u}{\Delta x_k} = A_k + \alpha_k = A_k.
$$
Таким образом, частная производная $\frac{\partial u}{\partial x_k}(M)$ существует для любого $k \in \{1,...,m \}$. 
$\blacktriangleright$


\textbf{Теорема.} Если функция $u = f(x_1, ..., x_m)$ дифференцируема в точке, то она непрерывна в этой точке.


$\blacktriangleleft$ Пусть
$$
	\Delta u = A_1\Delta x_1 + ... + A_m \Delta x_m + \alpha_1 \Delta x_1 + ... + \alpha_m \Delta x_m,
$$
где $A_i$ --- некоторые числа, $\alpha_i = \alpha_i(\Delta x_1, ..., \Delta x_m) \rightarrow 0$ при $ \forall \Delta x_i \rightarrow 0$, $\alpha_i = 0$ при $\Delta x_1 = ... = \Delta x_m = 0$.
Тогда 
$$
	\lim_{ \forall \Delta x_i \rightarrow 0} \Delta u = \lim_{\forall \Delta x_i \rightarrow 0} (A_1\Delta x_1 + ... + A_m \Delta x_m + \alpha_1 \Delta x_1 + ... + \alpha_m \Delta x_m) = 0.
$$
Таким образом,
$$
\lim_{\forall \Delta x_i \rightarrow 0} \Delta u = 0,
$$
что соответствует разностной форме условия непрерывности функции в точке. $\blacktriangleright$


\textbf{Замечение.} У функции могут существовать производные по всем переменным, но при этом она не будет дифференцируема в точке. Пример:
\begin{equation*}
f(x,y) = 
 \begin{cases}
   1 &\text{на осях координат}\\
   0 &\text{иначе.}
 \end{cases}
\end{equation*}

Частные производные $\frac{\partial f}{\partial x}(0,0) = \frac{\partial f}{\partial y}(0,0) = 0$, при этом функция не является непрерывной в точке (0,0), а значит не дифференцируема в ней. \\

\textbf{Теорема. Достаточное условие дифференцируемости функции.} Если функция $u = f(x_1, ..., x_m)$ имеет частные производные по всем переменным в некоторой $\varepsilon$-окрестности точки $M(x_1, ..., x_m)$, причем в самой точке $M$ эти частные производные непрерывны, то функция дифференцируема в точке $M$.

$\blacktriangleleft$ Проведем доказательство для функции двух переменных $u = f(x,y)$. Пусть частные производные $f'_x, f'_y$ существуют в $\varepsilon$-окрестности точки $M(x,y)$ и непрерывны в самой точке $M$.


Возьмем $\Delta x, \Delta y$ столь малыми, чтобы точка $M_1(x + \Delta x, y + \Delta y)$ лежала в $\varepsilon$-окрестности точки $M$ (см. Рис. 5). Тогда полное приращение функции в точке $M$ есть
$$	
	\Delta u = f(x + \Delta x, y + \Delta y) - f(x,y) = 
$$ 
$$
	= [f(x + \Delta x, y + \Delta y) - f(x, y + \Delta y)] + [f(x, y + \Delta y) - f(x,y)].
$$

\begin{figure}[h]
\centering
\includegraphics[width=0.5\linewidth]{diffM.png}
\caption{$\varepsilon$-окрестность точки $M$}
\label{fig:mpr}
\end{figure}

Заметим, что первая скобка соответствует частному приращению функции в точке $M_2$ при приращении $\Delta x$, а вторая скобка --- частному приращению функции в точке $M$ при приращении $\Delta y$. Так как функция $u = f(x,y)$  --- дифференцируема в $\varepsilon$-окрестности и непрерывна во всех рассматриваемых точках, то можно воспользоваться теоремой Лагранжа для разностей в скобках:
$$
f(x + \Delta x, y + \Delta y) - f(x, y + \Delta y) = f'_x(M_3) \Delta x,
$$
$$
	f(x, y + \Delta y) - f(x,y) = f'_y(M_4) \Delta y.
$$
Здесь $M_3 = (x + \theta_1 \Delta x, \Delta y)$, $M_4 = (x, y + \theta_2 \Delta y)$, $0 < \theta_i < 1$.


Так как производные $f'_x$, $f'_y$ непрерывны в точке $M$, то

$$
	\lim_{\Delta x,y \rightarrow 0} f'_x(x + \theta_1 \Delta x, y + \Delta y) =f'_x(x,y),
$$
$$
	\lim_{\Delta x,y \rightarrow 0} f'_y(x, y + \theta_2 \Delta y) = f'_y(x,y).
$$
Распишем приращение функции $u = f(x,y)$ в точке через производные  
$$
	\Delta u = [f'_x(x,y)\Delta x + \alpha_1 \Delta x] + [f'_y(x,y) \Delta y + \alpha_2 \Delta y] = 
$$
$$
	= f'_x(x,y)\Delta x + f'_y(x,y) \Delta y + \alpha_1 \Delta x + \alpha_2 \Delta y. 
$$
здесь $\alpha_{1,2} \rightarrow 0$ при $\Delta x, \Delta y \rightarrow 0$, $\alpha_1 = \alpha_2 = 0$ при $\Delta x = \Delta y = 0$.
Таким образом, мы записали приращение функции в необходимом виде, а значит она дифференцируема в точке $M$. $\blacktriangleright$ \\



\textbf{Дифференцируемость сложной фунцкии}

Рассмотрим сложную функцию $z=f(x,y)$, где $x=\varphi(u,v)$, $y = \psi(u,v)$.

\textbf{Теорема.} Пусть: \begin{enumerate}
\item функции $x=\varphi(u,v)$, $y = \psi(u,v)$ дифференцируемы в точке $(u_0, v_0)$,
\item функция $z=f(x,y)$ дифференцируема в точке $(x_0, y_0)$, где 
$x_0 = \varphi(u_0, v_0)$, $y_0 = \psi(u_0, v_0)$.
\end{enumerate}
Тогда функция $z=f(x,y) = f(\varphi(u,v), \psi(u,v))$ дифференцируема в точке $(u_0, v_0)$.

$\blacktriangleleft$ Распишем первое условие:
$$
	\Delta x = \frac{\partial \varphi}{\partial u} (u_0, v_0) \Delta u + \frac{\partial \varphi}{\partial v} (u_0,v_0) \Delta v + \alpha_1 \Delta u + \alpha_2 \Delta v,
	\eqno(*)
$$
$$
	\Delta y = \frac{\partial \psi}{\partial u} (u_0, v_0) \Delta u + \frac{\partial \psi}{\partial v} (u_0,v_0) \Delta v + \beta_1 \Delta u + \beta_2 \Delta v,
$$
где $\alpha_i, \beta_i \rightarrow 0$ при $\Delta u, \Delta v \rightarrow 0$, $\alpha_i = \beta_i = 0$ при $\Delta u = \Delta v = 0$.

При таких приращениях имеется приращение функции $z=f(x,y):$
$$
	\Delta z = \frac{\partial f}{\partial x} (x_0, y_0) \Delta x + \frac{\partial f}{\partial y} (x_0, y_0) \Delta y + \gamma_1 \Delta x + \gamma_2 \Delta y
$$
с соответствующими условиями на $\gamma_i$. 

Подставим $(*)$ в последнее равенство, получим
$$	
	\Delta z = A\Delta u + B\Delta v + \alpha \Delta u + \beta \Delta v, \eqno(**)
$$
где 
$$
	A = \frac{\partial f}{\partial x}(x_0, y_0)\cdot \frac{\varphi}{\partial u}(u_0, v_0) + \frac{\partial f}{\partial y}(x_0, y_0) \cdot \frac{\partial \psi}{\partial u} (u_0, v_0).
$$
Значения $B, \alpha, \beta$ расписываются аналогично.

Равенство (**) означает, что сложная функция $z = f(x,y)$ дифференцируема в точке $(u_0, v_0)$. $\blacktriangleright$


В процессе доказательства теоремы были получены формулы для производной сложной функции:
$$
	\frac{\partial z}{\partial u} = \frac{\partial z}{\partial x} \cdot \frac{\partial x}{\partial u} + \frac{\partial z}{\partial y} \cdot \frac{\partial y}{\partial u},  \quad
	\frac{\partial z}{\partial v} = \frac{\partial z}{\partial x} \cdot \frac{\partial x}{\partial v} + \frac{\partial z}{\partial y} \cdot \frac{\partial y}{\partial v}.
$$








\textbf{Дифференциал функции многих переменных}

Пусть функция $u = f(x_1, ..., x_m)$ дифференцируема в точке $M$, тогда ее приращение в этой точке можно представить в виде 
$$
	\Delta u = \left ( \frac{\partial u}{\partial x_1}(M) \Delta x_1 +...+
	\frac{\partial u}{\partial x_m}(M) \Delta x_m \right ) + (\alpha_1 \Delta x_1 +...+ \alpha_m \Delta x_m),
$$
где $\alpha_i \rightarrow 0$ при $\{\Delta x_1 \rightarrow 0, ..., \Delta x_m \rightarrow 0 \}$, $\alpha_i = 0$ при $x_1 = ... =x_m = 0$.

Обе суммы в скобках являются б.м.ф. при $\{\Delta x_1 \rightarrow 0, ..., \Delta x_m \rightarrow 0 \}$. При этом первая сумма является линейной относительно $\Delta_i$ частью приращения функции, а вторая сумма --- бесконечно малая более высокого порядка, чем линейная часть.


\textbf{Определение.} \textit{Дифференциалом} (\textit{полным дифференциалом}) функции $u = f(x_1,...,x_m)$ в точке $M$ называется линейная относительно $\Delta x_1, ..., \Delta x_m$ часть приращения функции в точке $M$:
$$
	du = \frac{\partial u}{\partial x_1}(M)\Delta x_1 + ... + \frac{\partial u}{\partial x_m}(M) \Delta x_m.
$$


Дифференциалом независимой переменной будем называть приращение этой переменной:
$$
	dx_i = \Delta x_i.
$$
В таком случае выражение дифференциала можно записать так:
$$	
	u = \frac{\partial u}{\partial x_1}(M) d x_1 + ... + \frac{\partial u}{\partial x_m}(M) d x_m. \eqno(\diamondsuit)
$$

\textbf{Замечание!} Использование частных производных объясняется тем, что наличие частных производных необходимо для дифференцируемости, а следовательно и для существования дифференциала. 


\textbf{Теорема. Об инвариантности формы первого дифференциала.}
Формула $(\diamondsuit)$ остается в силе, если $x_i$ являются не независимыми переменными, а дифференцируемыми функциями каких-то независимых переменных. \newpage


	\noindent\rule{\textwidth}{1pt}
	\textbf{8. (SE)} Достаточные условия дифференцируемости функции в точке. Теорема Лагрнажа о среднем (формула конечных приращений).
	
	
	\noindent\rule{\textwidth}{1pt}
	
	
	\textbf{Теорема. Необходимое и достаточное условие дифференцируемости функции одной переменной.} Функция одной переменной $f(x)$ дифференцируема в точке $x_0$ тогда и только тогда, когда существует производная $f'(x_0)$. Доказательство см. выше. \\
	
	\textbf{Теорема. Достаточное условие дифференцируемости функции.} Если функция $u = f(x_1, ..., x_m)$ имеет частные производные по всем переменным в некоторой $\varepsilon$-окрестности точки $M(x_1, ..., x_m)$, причем в самой точке $M$ эти частные производные непрерывны, то функция дифференцируема в точке $M$. Доказательство см. выше. \\
	
\textbf{Основные теоремы о дифференцируемых функциях}

\textbf{Теорема (Ферма).} Пусть функция $f(x)$ определена и дифференцируема в окрестности точки $a$. Если $a$ --- точка локального минимума (максимума), то $f'(a) = 0$.

\textit{Примечание.} Точка $a$ называется точкой \textit{локального минимума} (\textit{максимума}), если для некоторой окрестности $U(a)$ выполняется $x \in U(a) \Rightarrow x \geq (\leq) f(a)$.

$\blacktriangleleft$ Пусть $a$ --- точка локального максимума функции $f(x)$. Пусть $x \in U(a), \; x < a$, тогда
$$
	\lim_{x \rightarrow a - 0 } \frac{f(x) - f(a)}{x - a} = f'(a) \geq 0.
$$
С другой стороны при $x \in U(a)$, $x > a$ имеем
$$
\lim_{x \rightarrow a + 0 } \frac{f(x) - f(a)}{x - a} = f'(a) \leq 0.
$$
Из неравенств $f'(a) \geq 0, \; f'(a) \leq 0$ заключаем, $f'(a) = 0. \blacktriangleright$

Для доказательства теоремы Ролля необходимо доказать 

\textbf{Теорему Вейерштрасса о максимальном значении.} Функция, непрерывная на отрезке, ограничена на нем. При этом на отрезке есть точка, где функция принимает максимальное (минимальное) значение.


$\blacktriangleleft$ Если функция $f: E \rightarrow \mathbb{R}$ непрерывна в точке $a$, то $f$ ограничена в некоторой окрестности $U_E(a)$. Действительно, по определению непрерывности
$$
\forall \varepsilon > 0 \; \exists \delta > 0: \; \forall x \in E \; (|x - a| < \delta) \; \Rightarrow (|f(x) - f(a)| < \varepsilon),
$$ 
значит функция $f(x)$ ограничена внутри этой окрестности значениями $f(a) - \varepsilon, \; f(a) + \varepsilon$. 

Пусть $f: E \rightarrow \mathbb{R}$ непрерывна на отрезке $[a;b]$. Рассмотрим произвольную точку $x \in [a;b]$. Так как функция $f$ непрерывна в точке $x$, то она ограничена внутри некоторой окрестности $U_E(x) = E \cap U(x)$ данной точки. Совокупность таких окрестностей для всех точек отрезка образует покрытие отрезка. Из этой совокупности можно извлечь конечное покрытие $U_E(x_1), ..., U_E(x_k)$ (по лемме о конечном покрытии). Поскольку внутри каждой окрестности $U_E(x_i)$ функция ограничена, то она ограничена на всем отрезке:
$$
\min \{ \min\{U_E(x_1)\}, ..., \min\{U_E(x_k)\}\} \leq f(x)\leq \max \{ \max\{ U_E(x_1)\}, ..., \max\{ U_E(x_k)\} \}.
$$

Ограниченность на отрезке установлена.

Пусть теперь $M = \sup f(x)$. Пусть в любой точке $x \in E \; (f(x) < M)$,  тогда функция $M - f(x)$ нигде на $E$ не обращается в нуль. Рассмотрим функцию 
$$
g(x) = \frac{1}{M - f(x)}.
$$
Она непрерывна на $E$, так как $M - f(x) \neq 0$. При этом она не ограничена на $E$, так как по определению точной верхней границы для любого $\varepsilon > 0$ существует значение $f(x)$ такое, что $\sup - f(x) < \varepsilon$. Таким образом, мы получили непрерывную на отрезке функцию, которая не ограничена сверху, что противоречит только что установленному утверждению. Значит для любой непрерывной на отрезке $[a;b]$ функции найдется точка $c \in [a;b]$ такая, что $f(c) = \sup_{x \in E} f(x). \blacktriangleright$

\textbf{Замечание!} Условие непрерывности на отрезке (компакте) важно потому, что мы опирались в доказательстве на возможность покрытия компакта конечным числом окрестностей. Для интервала данная теорема не работает. Пример $f(x) = x$. На интервале $(0,1)$ эта функция не имеет ни минимального, ни максимального значений.  \\


\textbf{Теорема (Ролль).} Пусть функция $f(x)$ непрерывна на отрезке $[a;b]$ и дифференцируема на интервале $(a;b)$, причем $f(a) = f(b)$, тогда найдется точка $c \in (a;b)$ такая, что $f'(c) = 0$.

$\blacktriangleleft$ Поскольку $f$ непрерывна на отрезке $[a;b]$ то по теореме Вейерштрасса о максимальном значении найдутся точки $x_m, \; x_M \in [a;b]$, в которых функция принимает свои минимальное и максимальное значения. Если $f(x_m) = f(x_M)$, то функция постоянна на отрезке, а значит ее производная в любой точке равна 0.

Пусть теперь $f(x_m) < f(x_M)$. По условию $f(a) = f(b)$, поэтому одна из точек $x_m, \; x_M$ обязана лежать внутри интервала $(a;b)$. Без ограничения общности будем считать, что $x_m \in (a,b)$. Так как функция дифференцируема на интервале $(a,b)$, то существует $f'(x_m)$, а по теореме Ферма $f'(x_m) = 0. \blacktriangleright$ \\

\textbf{Теорема Лагранжа о конечном приращении.} Пусть функция $f$ непрерывна на $[a;b]$ и дифференцируема на $(a;b)$. Тогда найдется точка $c \in (a;b)$ такая, что
$$
	f(b) - f(a) = f'(c)(b-a).
$$

$\blacktriangleleft$ Рассмотрим вспомогательную функцию 
$$
	g(x) = f(x) - \frac{f(b) - f(a)}{b-a}(x-a),
$$
которая непрерывна на $[a,b]$ как сумма непрерывных функций и дифференцируема на $(a,b)$ как сумма дифференцируемы функций. При этом $g(a) = g(b) = f(a)$. Таким образом, $g(x)$ удовлетворяет условиям теоремы Ролля, значит существует точка $c \in (a;b)$ такая, что $g'(c) = 0$. При этом
$$
g'(x) = f'(x) - \frac{f(b) - f(a)}{b-a},
$$
значит 
$$
0 = g'(c) = f'(c) - \frac{f(b) - f(a)}{b-a}. \blacktriangleright
$$

\textbf{Следствие теоремы Лагранжа.} Если производная функции $f(x)$ в каждой точке интервала неотрицательна, то функция не убывает на этом интервале. $\blacktriangleleft$ Пусть $x_1, \, x_2 \in (a;b), \; x_1 < x_2$. Тогда 
$$	
f(x_2) - f(x_1) = f'(c)(x_2 - x_1),
$$
где $c \in (x_1, x_2)$, $f'(c) \geq 0$. Значит $f(x_2) \geq f(x_1)$.
$\blacktriangleright$


Справедлива также формула для приращений функции многих переменных
$$
f(x_1 + h_1, x_2 + h_2, ..., x_n + h_n) - f(x_1, x_2, ..., x_n) =
$$
$$
 = \sum_{i = 1}^n\frac{\partial f}{\partial x_i}(x_1 + \theta h_i),
$$
где $0 < \theta < 1$, $h_i$ --- приращения аргументов, $\partial f/ \partial x_i$ --- частная производная функции $f$ по переменной $x_i$. 
	
	

\end{document}










